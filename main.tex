\section{Mapzen}
Mapzen Merupakan laboratorium pemetaan open source dibawah Samsung Accelatoryang membangun dan mendukung data dan software terbuka untuk mempromosikan ekosistem pemetaan yang sehat. Tim Pengembangan dari Mapzen merupakan salah satu divisi dari SRA (Samsung Research Amerika) yang berfokus pada komponen inti dari platform geo, termasuk pencarian, rendering, navigasi, dan data.

Keunggulan dari produk Mapzen adalah penggunaan mesin rendering fleksibel yang disebut juga Tangram yang digunakan untuk menampilkan peta digital dalam nuansa 2D maupun 3D secara real-time. Mapzen dibangun dari berbagai peralatan open-source yang dikemas ke dalam layanan Web dan di hosting di server Mapzen.

\subsection{QGIS}
QGIS adalah perangkat Sistem Informasi Geografis (SIG) Open Source yang user friendly dengan lisensi di bawah GNU General Public License. QGIS merupakan proyek tidak resmi dari Open Source Geospatial Foundation (OSGeo). QGIS dapat dijalankan pada Linux, Unix, Mac OSX, Windows dan Android, serta mendukung banyak format dan fungsionalitas data vektor, raster, dan basisdata.


\subsection{Menu Pada QGIS}
source type ini berupa pilihan :
File | Directory | Database | Protocol
Protocol yang disupport adalah URI (Uniform Resources Identifier) untuk GeoJSON, GeoJSON sendiri merupakan enkoding open format data geografis bertipe JavaScript Open Notation.
Database yang disupport antara lain ESRI Personal Geodatabase, ODBC, Postgre, dan MySQL, artinya anda bahkan bisa buka Geodatabase ESRI walaupun kemungkinan ada beberapa format data khusus yang hilang, tetapi spatial-nya akan tetap dapat terbaca.


\subsection{file Format}
{
"type": "Feature",
"properties": {
    "osm_id": 368395980,
    "access": null,
    "aerialway": null,
    "aeroway": "helipad",
    "amenity": null,
    "area": null,
    "barrier": null,
    "bicycle": null,
    "brand": null,
    "bridge": null,
    "boundary": null,
    "building": null,
    "capital": null,
    "covered": null,
    "culvert": null,
    "cutting": null,
    "disused": null,
    "ele": "33",
    "embankment": null,
    "foot": null,
    "harbour": null,
    "highway": null,
    "historic": null,
    "horse": null,
    "junction": null,
    "landuse": null,
    "layer": null,
    "leisure": null,
    "lock": null,
    "man_made": null,
    "military": null,
    "motorcar": null,
    "name": "Unisys Heliport",
    "natural": null,
    "oneway": null,
    "operator": null,
    "poi": null,
    "population": null,
    "power": null,
    "place": null,
    "railway": null,
    "ref": null,
    "religion": null,
    "route": null,
    "service": null,
    "shop": null,
    "sport": null,
    "surface": null,
    "toll": null,
    "tourism": null,
    "tower:type": null,
    "tunnel": null,
    "water": null,
    "waterway": null,
    "wetland": null,
    "width": null,
    "wood": null,
    "z_order": null
},
"geometry": {
    "type": "Point",
    "coordinates": [
        -74.50099,
        40.3709408
    ]
}
}

/subsection{Panel Daftar Layer}
    Pada bagian kiri QGIS terdapat panel daftar layer. Daftar layer-layer ini, atau file, yang dimasukkan ke dalam proyek QGIS. Pada proyek ini, kita memliki dua layer.
    Panel layer tidak hanya menunjukkan semua file yang kita buka, tetapi juga menjelaskan susunan urutan yang akan kita gambar pada map canvas. Sebuah layer yang terdapat di posisi paling bawah akan tergambar pertama kali dan beberapa layer yang terdapat diatasnya akan digambar pada posisi paling atas.
    Salah satu format file umum adalah shapefiles, dengan ekstensi akhir dari nama file tersebut bertuliskan .shp. Shapefiles sering digunakan untuk menyimpan geodata, dan umumnya digunakan pada aplikasi SIG seperti QGIS.

/subsection{Membuat Layer}
    Dalam Quantum GIS, kita dapat membuat layer baru yang berupa Shapefile, layer Spatiallite, dan juga GPX layer. Contohnya, bagaimana membuat sebuah Shapefile layer di Quantum GIS. 
    Langkah pertama buka Quantum GIS dan buatlah sebuah project baru. Kemudian klik Layer | Create Layer dan pilih New Shapefile layer. Quantum GIS mendukung  tiga tipe object yaitu Point (titik), Line (garis) dan juga Polygon. Kita harus memilih salah satu diantara tiga tersebut. Jadi dalam satu buah layer, kita tidak bisa mencampurkan ketiga buah tipe object tersebut. Dalam contoh kali ini bagaimana untuk membuat layer point. Pilih Point pada bagian Type. Untuk file encoding, kita gunakan System sebagai nilai defaultnya. Jangan lupa pula untuk menentukan proyeksi layer anda. Dalam contoh ini saya menggunakan WGS 84.  
    Kemudian kita perlu menambahkan atribut tabel, yang berupa kolom untuk menampung informasi dalam sebuah layer.
    Tekan OK dan tentukan nama file dan lokasi untuk menyimpan file SHP nya. Kemudian layer kita sudah siap digunakan. Anda dapat mulai digitasi point dengan mengaktifkan mode editing di toolbar. 
